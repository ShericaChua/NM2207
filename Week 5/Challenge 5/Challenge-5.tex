% Options for packages loaded elsewhere
\PassOptionsToPackage{unicode}{hyperref}
\PassOptionsToPackage{hyphens}{url}
%
\documentclass[
]{article}
\usepackage{amsmath,amssymb}
\usepackage{iftex}
\ifPDFTeX
  \usepackage[T1]{fontenc}
  \usepackage[utf8]{inputenc}
  \usepackage{textcomp} % provide euro and other symbols
\else % if luatex or xetex
  \usepackage{unicode-math} % this also loads fontspec
  \defaultfontfeatures{Scale=MatchLowercase}
  \defaultfontfeatures[\rmfamily]{Ligatures=TeX,Scale=1}
\fi
\usepackage{lmodern}
\ifPDFTeX\else
  % xetex/luatex font selection
\fi
% Use upquote if available, for straight quotes in verbatim environments
\IfFileExists{upquote.sty}{\usepackage{upquote}}{}
\IfFileExists{microtype.sty}{% use microtype if available
  \usepackage[]{microtype}
  \UseMicrotypeSet[protrusion]{basicmath} % disable protrusion for tt fonts
}{}
\makeatletter
\@ifundefined{KOMAClassName}{% if non-KOMA class
  \IfFileExists{parskip.sty}{%
    \usepackage{parskip}
  }{% else
    \setlength{\parindent}{0pt}
    \setlength{\parskip}{6pt plus 2pt minus 1pt}}
}{% if KOMA class
  \KOMAoptions{parskip=half}}
\makeatother
\usepackage{xcolor}
\usepackage[margin=1in]{geometry}
\usepackage{color}
\usepackage{fancyvrb}
\newcommand{\VerbBar}{|}
\newcommand{\VERB}{\Verb[commandchars=\\\{\}]}
\DefineVerbatimEnvironment{Highlighting}{Verbatim}{commandchars=\\\{\}}
% Add ',fontsize=\small' for more characters per line
\usepackage{framed}
\definecolor{shadecolor}{RGB}{248,248,248}
\newenvironment{Shaded}{\begin{snugshade}}{\end{snugshade}}
\newcommand{\AlertTok}[1]{\textcolor[rgb]{0.94,0.16,0.16}{#1}}
\newcommand{\AnnotationTok}[1]{\textcolor[rgb]{0.56,0.35,0.01}{\textbf{\textit{#1}}}}
\newcommand{\AttributeTok}[1]{\textcolor[rgb]{0.13,0.29,0.53}{#1}}
\newcommand{\BaseNTok}[1]{\textcolor[rgb]{0.00,0.00,0.81}{#1}}
\newcommand{\BuiltInTok}[1]{#1}
\newcommand{\CharTok}[1]{\textcolor[rgb]{0.31,0.60,0.02}{#1}}
\newcommand{\CommentTok}[1]{\textcolor[rgb]{0.56,0.35,0.01}{\textit{#1}}}
\newcommand{\CommentVarTok}[1]{\textcolor[rgb]{0.56,0.35,0.01}{\textbf{\textit{#1}}}}
\newcommand{\ConstantTok}[1]{\textcolor[rgb]{0.56,0.35,0.01}{#1}}
\newcommand{\ControlFlowTok}[1]{\textcolor[rgb]{0.13,0.29,0.53}{\textbf{#1}}}
\newcommand{\DataTypeTok}[1]{\textcolor[rgb]{0.13,0.29,0.53}{#1}}
\newcommand{\DecValTok}[1]{\textcolor[rgb]{0.00,0.00,0.81}{#1}}
\newcommand{\DocumentationTok}[1]{\textcolor[rgb]{0.56,0.35,0.01}{\textbf{\textit{#1}}}}
\newcommand{\ErrorTok}[1]{\textcolor[rgb]{0.64,0.00,0.00}{\textbf{#1}}}
\newcommand{\ExtensionTok}[1]{#1}
\newcommand{\FloatTok}[1]{\textcolor[rgb]{0.00,0.00,0.81}{#1}}
\newcommand{\FunctionTok}[1]{\textcolor[rgb]{0.13,0.29,0.53}{\textbf{#1}}}
\newcommand{\ImportTok}[1]{#1}
\newcommand{\InformationTok}[1]{\textcolor[rgb]{0.56,0.35,0.01}{\textbf{\textit{#1}}}}
\newcommand{\KeywordTok}[1]{\textcolor[rgb]{0.13,0.29,0.53}{\textbf{#1}}}
\newcommand{\NormalTok}[1]{#1}
\newcommand{\OperatorTok}[1]{\textcolor[rgb]{0.81,0.36,0.00}{\textbf{#1}}}
\newcommand{\OtherTok}[1]{\textcolor[rgb]{0.56,0.35,0.01}{#1}}
\newcommand{\PreprocessorTok}[1]{\textcolor[rgb]{0.56,0.35,0.01}{\textit{#1}}}
\newcommand{\RegionMarkerTok}[1]{#1}
\newcommand{\SpecialCharTok}[1]{\textcolor[rgb]{0.81,0.36,0.00}{\textbf{#1}}}
\newcommand{\SpecialStringTok}[1]{\textcolor[rgb]{0.31,0.60,0.02}{#1}}
\newcommand{\StringTok}[1]{\textcolor[rgb]{0.31,0.60,0.02}{#1}}
\newcommand{\VariableTok}[1]{\textcolor[rgb]{0.00,0.00,0.00}{#1}}
\newcommand{\VerbatimStringTok}[1]{\textcolor[rgb]{0.31,0.60,0.02}{#1}}
\newcommand{\WarningTok}[1]{\textcolor[rgb]{0.56,0.35,0.01}{\textbf{\textit{#1}}}}
\usepackage{graphicx}
\makeatletter
\def\maxwidth{\ifdim\Gin@nat@width>\linewidth\linewidth\else\Gin@nat@width\fi}
\def\maxheight{\ifdim\Gin@nat@height>\textheight\textheight\else\Gin@nat@height\fi}
\makeatother
% Scale images if necessary, so that they will not overflow the page
% margins by default, and it is still possible to overwrite the defaults
% using explicit options in \includegraphics[width, height, ...]{}
\setkeys{Gin}{width=\maxwidth,height=\maxheight,keepaspectratio}
% Set default figure placement to htbp
\makeatletter
\def\fps@figure{htbp}
\makeatother
\setlength{\emergencystretch}{3em} % prevent overfull lines
\providecommand{\tightlist}{%
  \setlength{\itemsep}{0pt}\setlength{\parskip}{0pt}}
\setcounter{secnumdepth}{-\maxdimen} % remove section numbering
\ifLuaTeX
  \usepackage{selnolig}  % disable illegal ligatures
\fi
\IfFileExists{bookmark.sty}{\usepackage{bookmark}}{\usepackage{hyperref}}
\IfFileExists{xurl.sty}{\usepackage{xurl}}{} % add URL line breaks if available
\urlstyle{same}
\hypersetup{
  pdftitle={Challenge-5},
  pdfauthor={Sherica Chua},
  hidelinks,
  pdfcreator={LaTeX via pandoc}}

\title{Challenge-5}
\author{Sherica Chua}
\date{13/09/2023}

\begin{document}
\maketitle

\hypertarget{questions}{%
\subsection{Questions}\label{questions}}

\hypertarget{question-1-local-variable-shadowing}{%
\paragraph{Question-1: Local Variable
Shadowing}\label{question-1-local-variable-shadowing}}

Create an R function that defines a global variable called \texttt{x}
with a value of 5. Inside the function, declare a local variable also
named \texttt{x} with a value of 10. Print the value of \texttt{x} both
inside and outside the function to demonstrate shadowing.

\textbf{Solutions:}

\begin{Shaded}
\begin{Highlighting}[]
\CommentTok{\# Enter code here}
\NormalTok{x}\OtherTok{\textless{}{-}}\DecValTok{5}
\NormalTok{foo}\OtherTok{\textless{}{-}} \ControlFlowTok{function}\NormalTok{() \{}
\NormalTok{  x}\OtherTok{\textless{}{-}}\DecValTok{10}
  \FunctionTok{print}\NormalTok{(x)}
\NormalTok{\}}
\FunctionTok{print}\NormalTok{(x)}
\end{Highlighting}
\end{Shaded}

\begin{verbatim}
## [1] 5
\end{verbatim}

\begin{Shaded}
\begin{Highlighting}[]
\FunctionTok{foo}\NormalTok{()}
\end{Highlighting}
\end{Shaded}

\begin{verbatim}
## [1] 10
\end{verbatim}

\hypertarget{question-2-modify-global-variable}{%
\paragraph{Question-2: Modify Global
Variable}\label{question-2-modify-global-variable}}

Create an R function that takes an argument and adds it to a global
variable called \texttt{total}. Call the function multiple times with
different arguments to accumulate the values in \texttt{total}.

\textbf{Solutions:}

\begin{Shaded}
\begin{Highlighting}[]
\CommentTok{\# Enter code here}
\NormalTok{x}\OtherTok{\textless{}{-}}\DecValTok{5}
\NormalTok{total}\OtherTok{\textless{}{-}}\DecValTok{4}
\NormalTok{add\_to}\OtherTok{\textless{}{-}}\ControlFlowTok{function}\NormalTok{(x)\{}
\NormalTok{  total}\OtherTok{\textless{}\textless{}{-}}\NormalTok{x}\SpecialCharTok{+}\NormalTok{total}
\NormalTok{\}}
\FunctionTok{add\_to}\NormalTok{(}\DecValTok{5}\NormalTok{)}
\FunctionTok{print}\NormalTok{(total)}
\end{Highlighting}
\end{Shaded}

\begin{verbatim}
## [1] 9
\end{verbatim}

\begin{Shaded}
\begin{Highlighting}[]
\FunctionTok{add\_to}\NormalTok{(}\DecValTok{10}\NormalTok{)}
\FunctionTok{print}\NormalTok{(total)}
\end{Highlighting}
\end{Shaded}

\begin{verbatim}
## [1] 19
\end{verbatim}

\hypertarget{question-3-global-and-local-interaction}{%
\paragraph{Question-3: Global and Local
Interaction}\label{question-3-global-and-local-interaction}}

Write an R program that includes a global variable \texttt{total} with
an initial value of 100. Create a function that takes an argument, adds
it to \texttt{total}, and returns the updated \texttt{total}.
Demonstrate how this function interacts with the global variable.

\textbf{Solutions:}

\begin{Shaded}
\begin{Highlighting}[]
\CommentTok{\# Enter code here}
\NormalTok{total}\OtherTok{\textless{}{-}}\DecValTok{100}
\NormalTok{add}\OtherTok{\textless{}{-}}\DecValTok{5}

\NormalTok{new\_total}\OtherTok{\textless{}{-}}\ControlFlowTok{function}\NormalTok{(add)\{}
\NormalTok{  total}\OtherTok{\textless{}\textless{}{-}}\NormalTok{total}\SpecialCharTok{+}\NormalTok{add}
\NormalTok{\}}
\FunctionTok{new\_total}\NormalTok{(}\DecValTok{5}\NormalTok{)}
\NormalTok{total}
\end{Highlighting}
\end{Shaded}

\begin{verbatim}
## [1] 105
\end{verbatim}

\hypertarget{question-4-nested-functions}{%
\paragraph{Question-4: Nested
Functions}\label{question-4-nested-functions}}

Define a function \texttt{outer\_function} that declares a local
variable \texttt{x} with a value of 5. Inside \texttt{outer\_function},
define another function \texttt{inner\_function} that prints the value
of \texttt{x}. Call both functions to show how the inner function
accesses the variable from the outer function's scope.

\textbf{Solutions:}

\begin{Shaded}
\begin{Highlighting}[]
\CommentTok{\# Enter code here}
\NormalTok{Outer\_func }\OtherTok{\textless{}{-}} \ControlFlowTok{function}\NormalTok{() \{}
\NormalTok{  x}\OtherTok{\textless{}{-}}\DecValTok{5}
\NormalTok{  Inner\_func }\OtherTok{\textless{}{-}} \ControlFlowTok{function}\NormalTok{() \{}
    \FunctionTok{print}\NormalTok{(x)}
\NormalTok{  \}}
  \FunctionTok{Inner\_func}\NormalTok{()}
\NormalTok{\}}
\FunctionTok{Outer\_func}\NormalTok{()}
\end{Highlighting}
\end{Shaded}

\begin{verbatim}
## [1] 5
\end{verbatim}

\hypertarget{question-5-meme-generator-function}{%
\paragraph{Question-5: Meme Generator
Function}\label{question-5-meme-generator-function}}

Create a function that takes a text input and generates a humorous meme
with the text overlaid on an image of your choice. You can use the
\texttt{magick} package for image manipulation. You can find more
details about the commands offered by the package, with some examples of
annotating images here:
\url{https://cran.r-project.org/web/packages/magick/vignettes/intro.html}

\textbf{Solutions:}

\begin{Shaded}
\begin{Highlighting}[]
\CommentTok{\# Enter code here}

\FunctionTok{library}\NormalTok{(magick)}
\end{Highlighting}
\end{Shaded}

\begin{verbatim}
## Linking to ImageMagick 6.9.12.93
## Enabled features: cairo, freetype, fftw, ghostscript, heic, lcms, pango, raw, rsvg, webp
## Disabled features: fontconfig, x11
\end{verbatim}

\begin{Shaded}
\begin{Highlighting}[]
\FunctionTok{str}\NormalTok{(magick}\SpecialCharTok{::}\FunctionTok{magick\_config}\NormalTok{())}
\end{Highlighting}
\end{Shaded}

\begin{verbatim}
## List of 24
##  $ version           :Class 'numeric_version'  hidden list of 1
##   ..$ : int [1:4] 6 9 12 93
##  $ modules           : logi FALSE
##  $ cairo             : logi TRUE
##  $ fontconfig        : logi FALSE
##  $ freetype          : logi TRUE
##  $ fftw              : logi TRUE
##  $ ghostscript       : logi TRUE
##  $ heic              : logi TRUE
##  $ jpeg              : logi TRUE
##  $ lcms              : logi TRUE
##  $ libopenjp2        : logi TRUE
##  $ lzma              : logi TRUE
##  $ pangocairo        : logi TRUE
##  $ pango             : logi TRUE
##  $ png               : logi TRUE
##  $ raw               : logi TRUE
##  $ rsvg              : logi TRUE
##  $ tiff              : logi TRUE
##  $ webp              : logi TRUE
##  $ wmf               : logi FALSE
##  $ x11               : logi FALSE
##  $ xml               : logi TRUE
##  $ zero-configuration: logi TRUE
##  $ threads           : int 1
\end{verbatim}

\begin{Shaded}
\begin{Highlighting}[]
\NormalTok{Meme }\OtherTok{\textless{}{-}} \ControlFlowTok{function}\NormalTok{(path) \{}
\NormalTok{  frink}\OtherTok{\textless{}{-}}\FunctionTok{image\_read}\NormalTok{(path)}
\FunctionTok{image\_annotate}\NormalTok{(frink, }\StringTok{"CONFIDENTIAL"}\NormalTok{, }\AttributeTok{size =} \DecValTok{30}\NormalTok{, }\AttributeTok{color =} \StringTok{"red"}\NormalTok{, }\AttributeTok{boxcolor =} \StringTok{"pink"}\NormalTok{,}
  \AttributeTok{degrees =} \DecValTok{60}\NormalTok{, }\AttributeTok{location =} \StringTok{"+50+100"}\NormalTok{)}
\NormalTok{\}}

\FunctionTok{Meme}\NormalTok{(}\StringTok{"https://jeroen.github.io/images/frink.png"}\NormalTok{)}
\end{Highlighting}
\end{Shaded}

\includegraphics{Challenge-5_files/figure-latex/unnamed-chunk-5-1.png}

\hypertarget{question-6-text-analysis-game}{%
\paragraph{Question-6: Text Analysis
Game}\label{question-6-text-analysis-game}}

Develop a text analysis game in which the user inputs a sentence, and
the R function provides statistics like the number of words, characters,
and average word length. Reward the user with a ``communication skill
level'' based on their input.

\textbf{Solutions:}

\begin{Shaded}
\begin{Highlighting}[]
\CommentTok{\# Enter code here}

\CommentTok{\# Function to calculate communication skill level}
\NormalTok{calculate\_skill\_level }\OtherTok{\textless{}{-}} \ControlFlowTok{function}\NormalTok{(avg\_word\_length) \{}
  \ControlFlowTok{if}\NormalTok{ (avg\_word\_length }\SpecialCharTok{\textgreater{}=} \DecValTok{6}\NormalTok{) \{}
    \FunctionTok{return}\NormalTok{(}\StringTok{"Advanced"}\NormalTok{)}
\NormalTok{  \} }\ControlFlowTok{else} \ControlFlowTok{if}\NormalTok{ (avg\_word\_length }\SpecialCharTok{\textgreater{}=} \DecValTok{4}\NormalTok{) \{}
    \FunctionTok{return}\NormalTok{(}\StringTok{"Intermediate"}\NormalTok{)}
\NormalTok{  \} }\ControlFlowTok{else}\NormalTok{ \{}
    \FunctionTok{return}\NormalTok{(}\StringTok{"Beginner"}\NormalTok{)}
\NormalTok{  \}}
\NormalTok{\}}

\CommentTok{\# Function to play the text analysis game}
\NormalTok{play\_text\_analysis\_game }\OtherTok{\textless{}{-}} \ControlFlowTok{function}\NormalTok{() \{}
  \FunctionTok{cat}\NormalTok{(}\StringTok{"Welcome to the Text Analysis Game!}\SpecialCharTok{\textbackslash{}n}\StringTok{"}\NormalTok{)}
  \FunctionTok{cat}\NormalTok{(}\StringTok{"Enter a sentence, and I will provide you with statistics and your communication skill level.}\SpecialCharTok{\textbackslash{}n}\StringTok{"}\NormalTok{)}
  
  \CommentTok{\# Get user input}
\NormalTok{  sentence }\OtherTok{\textless{}{-}} \FunctionTok{readline}\NormalTok{(}\AttributeTok{prompt =} \StringTok{"Enter a sentence: "}\NormalTok{)}
  
  \CommentTok{\# Calculate statistics}
\NormalTok{  words }\OtherTok{\textless{}{-}} \FunctionTok{unlist}\NormalTok{(}\FunctionTok{strsplit}\NormalTok{(sentence, }\StringTok{" "}\NormalTok{))}
\NormalTok{  num\_words }\OtherTok{\textless{}{-}} \FunctionTok{length}\NormalTok{(words)}
\NormalTok{  num\_characters }\OtherTok{\textless{}{-}} \FunctionTok{nchar}\NormalTok{(sentence)}
\NormalTok{  avg\_word\_length }\OtherTok{\textless{}{-}} \FunctionTok{mean}\NormalTok{(}\FunctionTok{nchar}\NormalTok{(words))}
  
  \FunctionTok{cat}\NormalTok{(}\StringTok{"}\SpecialCharTok{\textbackslash{}n}\StringTok{{-}{-}{-} Text Statistics {-}{-}{-}}\SpecialCharTok{\textbackslash{}n}\StringTok{"}\NormalTok{)}
  \FunctionTok{cat}\NormalTok{(}\StringTok{"Number of words:"}\NormalTok{, num\_words, }\StringTok{"}\SpecialCharTok{\textbackslash{}n}\StringTok{"}\NormalTok{)}
  \FunctionTok{cat}\NormalTok{(}\StringTok{"Number of characters:"}\NormalTok{, num\_characters, }\StringTok{"}\SpecialCharTok{\textbackslash{}n}\StringTok{"}\NormalTok{)}
  \FunctionTok{cat}\NormalTok{(}\StringTok{"Average word length:"}\NormalTok{, avg\_word\_length, }\StringTok{"}\SpecialCharTok{\textbackslash{}n}\StringTok{"}\NormalTok{)}
  
  \CommentTok{\# Calculate and display communication skill level}
\NormalTok{  skill\_level }\OtherTok{\textless{}{-}} \FunctionTok{calculate\_skill\_level}\NormalTok{(avg\_word\_length)}
  \FunctionTok{cat}\NormalTok{(}\StringTok{"}\SpecialCharTok{\textbackslash{}n}\StringTok{{-}{-}{-} Communication Skill Level {-}{-}{-}}\SpecialCharTok{\textbackslash{}n}\StringTok{"}\NormalTok{)}
  \FunctionTok{cat}\NormalTok{(}\StringTok{"Your communication skill level is:"}\NormalTok{, skill\_level, }\StringTok{"}\SpecialCharTok{\textbackslash{}n}\StringTok{"}\NormalTok{)}
\NormalTok{\}}

\CommentTok{\# Play the game}
\FunctionTok{play\_text\_analysis\_game}\NormalTok{()}
\end{Highlighting}
\end{Shaded}


\end{document}
